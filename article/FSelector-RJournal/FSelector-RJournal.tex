% !TeX root = RJwrapper.tex
\title{Rcpp Implementation of Entropy Based Feature Selection Algorithms with
Sparse Matrix Support}
\author{by Zygmunt Zawadzki, Marcin Kosiński}

\maketitle

\abstract{%
\par 

Feature selection is a process of extracting valuable features that have
significant influence on dependent variable. Time efficient feature
selection algorithms are still an active field of research and are in
the high demand in the machine learning area. \par We introduce
\pkg{FSelectorRcpp}, an \R ~package \citep{R} that includes entropy
based feature selection algorithms. Methods presented in this package
are not new, they were reimplemented in C++ and originally come from
\pkg{FSelector} package \citep{FSelector}. Our reimplementation occures
to have shorter computation times, it does not require earlier Java nor
Weka \citep{Hall:2009:WDM:1656274.1656278} installation and provides
support for sparse matrix format of data, e.g.~presented in \pkg{Matrix}
package \citep{Matrix}. This approach facilitates software installation
and improves work with bigger datasets, in comparison to the base
\R ~implementation in \pkg{FSelector}, which is even not optimal in the
sense of \R ~code. \par Additionally, we present new, C++ implementation
of continuous variables Multi-Interval Discretization (MDL) method
\citep{Fayyad1993}, which is required in entropy calculations during the
feature selection process in showed methods. By default, regular
\pkg{FSelector} implementation uses \pkg{entropy} package
\citep{entropy}, for which we also attach the computation times
comparison. \par Finally, we announce the full list of available
functions, which are divided to 2 groups: entropy based feature
selection methods and stepwise attribute selection functions that might
use any evaluator to choose propoer features, e.g.~presented entropy
based algorithms.
}

\section{Introduction and Motivation}\label{introduction-and-motivation}

\section{}\label{section}

\bibliography{RJreferences}

\address{%
Zygmunt Zawadzki\\
\\
\\
}
\href{mailto:zygmunt@zstat.pl}{\nolinkurl{zygmunt@zstat.pl}}

\address{%
Marcin Kosiński\\
Warsaw Univeristy of Technology\\
Faculty of Mathematics and Information Science\\ Koszykowa 75, Warsaw Poland\\
}
m.kosinski\R@mini.pw.edu.pl

